\documentclass[25pt]{tikzposter}

% design based on https://twitter.com/mikemorrison/status/1110191245035479041?lang=en
% and https://github.com/SimonLarsen/tikzpostersdu

\usepackage{tikzposterSDU}
\usepackage{bm}
\usepackage{amsfonts}       % blackboard math symbols
\usepackage{amsmath}
\usepackage{pgfplots}

\geometry{paperwidth=90cm, paperheight=122cm}

% metropolis colors
\definecolor{mDarkTeal}{HTML}{23373b}
\definecolor{mDarkBrown}{HTML}{604c38}
\definecolor{mLightBrown}{HTML}{EB811B}
\definecolor{mLightGreen}{HTML}{14B03D}

\usetheme{SDU}
\usepackage[sfdefault]{FiraSans}

% background color
\colorlet{blockbodybgcolor}{white}
\colorlet{backgroundcolor}{mDarkTeal}

% custom commands
\newcommand{\half}{\frac{1}{2}}
\newcommand{\x}{\bm{x}}
\newcommand{\xh}{\hat{\bm{x}}}
\newcommand{\e}{\bm{e}}
\newcommand{\z}{\bm{z}}
\newcommand{\mz}{\bm{\mu}_{z}}
\newcommand{\w}{\bm{w}}
\newcommand{\bo}{\bm{b}}
\newcommand{\A}{\bm{A}}

\newcommand{\se}{\sigma_e}
\newcommand{\sz}{\bm{\sigma_z}}
\newcommand{\laz}{\bm{\lambda}_z}

\newcommand{\X}{\bm{X}}
\newcommand{\Z}{\bm{Z}}
\newcommand{\N}{\mathcal{N}}
\newcommand{\U}{\mathcal{U}}
\newcommand{\E}[2]{\text{E}_{#1}\left[#2\right]}
\newcommand{\KL}{\text{KL}}


\begin{document}

% remove offset that would otherwise be fixed by \maketitle
\makeatletter
    \setlength{\TP@blocktop}{.47\textheight}
\makeatother

%% MAIN MESSAGE OF PAPER %%
\colorlet{blockbodybgcolor}{mDarkTeal}
\block{}{
  \color{white}{
    \fontseries{l}\fontsize{130}{100}\selectfont
    Learning \textbf{physical concepts}  purely \\\\\\
    from data: We demonstrate how \\\\\\
    \textbf{generative models} can learn  \\\\\\
    \textbf{manifolds} of \textbf{differential equations.}
  }
}
\colorlet{blockbodybgcolor}{white}


\begin{columns}
  \column{1.0}
  \block{}{
    %% TITLE OF PAPER %%
    {\Huge\bf Rodent: Relevance determination in ODE \\\\}
    {\Large Niklas Heim, V\'aclav \v Sm\'idl, Tom\'a\v s Pevn\'y} \hfill
    {\Large Czech Technical University, Prague}
  }

  \block{}{
    \begin{center}
    {\huge\bf Learning differential equations}
    \end{center}
    \Large
    \begin{minipage}{.28\textwidth}
      \innerblock{}{
      \begin{itemize}
        \item We want to find the simplest ODE that describes a dynamical
          system
      \end{itemize}
      }
    \end{minipage}
    \hspace{.01\textwidth}
    \begin{minipage}{.28\textwidth}
      \innerblock{}{
        \begin{itemize}
          \item Simple means: minimal order of ODE \& minimum No. of
            non-zero parameters.
        \end{itemize}
      }
    \end{minipage}
    \hspace{.01\textwidth}
    \begin{minipage}{.28\textwidth}
      \innerblock{}{
        \begin{itemize}
        \item Discover physically meaningful Eq.
          to help understand the underlying process.
        %\item We can learn manifolds of generating models not only a single process
        \end{itemize}
      }
    \end{minipage}

    \begin{tikzfigure}
      \includegraphics[width=.60\textwidth]{rodent.pdf}
    \end{tikzfigure}

    \vspace{1cm}
    \begin{center}
      {\huge\bf Advantages of the relevant ODE identifier}\\
    \end{center}

    \vspace{1cm}
    \begin{minipage}[t]{.28\textwidth}
    \Large
    \begin{itemize}
      \item \textbf{Explainability.}  Parameters of $\bm z$ are
        decoded through ODE solver, giving them physical meaning.
    \end{itemize}
    \end{minipage}
    \hspace{.01\textwidth}
    \begin{minipage}[t]{.28\textwidth}
    \begin{itemize}
      \item \textbf{Sparsity.} The ARD prior on
        $\bm{z}$ encourages the simplest solution with fewest non-zero parameters.
    \end{itemize}
    \end{minipage}
    \hspace{.01\textwidth}
    \begin{minipage}[t]{.28\textwidth}
    \begin{itemize}
      \item \textbf{Partial observations.} Rodent allows learning of an ODE
        without knowledge of all state trajectories. 
    \end{itemize}
    \end{minipage}



    \vspace{1cm}
    \begin{center}
      {\huge\bf Manifold learning \& Reidentification}\\
    \end{center}
    \begin{tikzfigure}
      \includegraphics[width=.60\textwidth]{single_enc_rec.pdf}
    \end{tikzfigure}
    \large
    Rodent reconstructions (left column) and latent codes (right column) of a
    harmonic signal in the upper plots. Reidentified reconstruction and
    encodings in the bottom. The heatmaps on the right show
    the corresponding encodings for the weights $\bm W$, biases $\bm b$, and
    initial conditions $\bm\xi$. The Rodent reduced
    the latent space to the four truly relevant parameters.
  }

  % \column{0.3}
  % \block{Rodent in depth}{
  %   Assume a time series $\X = [\bm x_1, \bm x_2,\ldots, \bm x_K]$ with $\bm x_i
  %   \in \mathbb{R}^d$ generated from discrete-time, noisy observations
  %   \begin{equation}
  %     \bm x_k = H(\bm{\xi}(\Delta t k)) + e_k,
  %   \end{equation}
  %   where $k=1\ldots K$, $e_k \sim \N(0,\se^2\mathbf{I})$, and partial obs.
  %   operator $H$.  The evolution of $\bm\xi(t) \in \mathbb{R}^N$ is governed by
  %   an ODE:
  %   \begin{align}
  %     \label{eq:dyn_sys}
  %     \frac{\partial \bm{\xi}}{\partial t} 
  %       = & f(\bm{\theta}, t) \approx \bm{W}\bm{\xi} + \bm{b}.
  %   \end{align}
  %   We aim to learn structure and order of the ODE from a set of
  %   trajectories $\{\X_i\}_{i=1}^{L}$  generated by the same generative process
  %   but with  different $\bm\theta_i$ and different $\bm\xi_i(0)$, for each
  %   trajectory, i.e.
  %   \begin{align}
  %     \label{eq:series_x}
  %     \X_i =& H(\psi(\bm\theta_i,\bm{\xi}_i(0), \bm{t}))+\bm{e},
  %   \end{align}
  %   where $\bm{t} = [0, \Delta t, \ldots, K\Delta t]$ and ODE
  %   solver $\psi$.  Assuming we observe a system with expected order $M$, we choose $N
  %   \geq M$.

  %   If ODE state and parameters are combined in $\z=[\bm\theta,\bm\xi(0)]$ the
  %   likelihood becomes
  %   \begin{align}
  %     \label{eq:decoder}
  %     p(\x|\z) &= \N(\x|H(\psi(\z)),\sigma_x^2).
  %   \end{align}
  %   To determine the structure of the ODE, we employ the ARD prior:
  %   \begin{align}
  %     \label{eq:ard_model}
  %     p(\z) &= \N(\z|0,\text{diag}(\laz^2)) &
  %     p(\laz) &= 1/\laz.%\Gamma(\laz|\alpha_0, \beta_0)
  %   \end{align}

  %   The posterior of $\bm z$ is prescribed by
  %   \begin{align}
  %     \label{eq:ard_posterior}
  %     p(\z|\x) = \N(\z|\phi_\omega(\x), \bm{\sigma}_z^2)
  %   \end{align}
  %   where mean $\mz=\phi_\omega(\x)$ is a NN with parameters
  %   $\omega$.
  %   The resulting ELBO:
  %   {\fontsize{25}{20}
  %   \begin{equation}
  %   \begin{aligned}
  %     \label{eq:elbo_reparam}
  %     \mathcal{L} &= \sum_{i=1}^n \E{p(z|x)}{\frac{(\x_i - \psi(\phi_\omega(\x_i) + \bm{\sigma}_z \odot \bm{\epsilon}))^2}{2\se^2}}
  %                 + \frac{nd}{2}\log(\se) \\
  %                 &+ \sum_{i=1}^n \left(
  %                     \log\left(\frac{\laz^2}{\sz^2}\right)
  %                     -m + \frac{\sz^2}{\laz^2} + \frac{\phi_\omega(\x_i)^2}{\laz^2}
  %                 \right),
  %   \end{aligned}
  %   \end{equation}}
  %   \hspace{-.6cm}with gaussian noise $\bm{\epsilon}$, decoder
  %   $H(\psi(\bm\theta,\bm\xi(0)))\equiv\phi(\z)$, and $\text{dim}(\z) = m$.
  %   The encoder network consists of two parts: (i) A dense network that receives
  %   only a few steps of the beginning of the time series, responsible for
  %   predicting $\bm \xi(0)$. (ii) A (CNN) that
  %   predicts $\bm \theta$.  The CNN averages over the time
  %   dimension after the convolutions, which makes it possible to use samples of
  %   different length.

  %   \textbf{Reidentification.} During reidentification we sample a batch of
  %   latent codes from the encoder for each input sample.  The latent samples
  %   are used as starting points for another optimization of the reconstruction
  %   error, while keeping all irrelevant parameters fixed.
  %   On the left, four parameters, namely $W_{13}$,
  %   $W_{31}$, $\xi_{1}$, and $\xi_{3}$, were found to be relevant, so only
  %   those are changing during the optimization with respect to $R$.
  %   This means that we stay in the identified model manifold, but are able to
  %   extrapolate far beyond the training range.   
  %   \vspace{0.1cm}
  % }
  
\end{columns}

\begin{columns}
  
\column{0.7}
\block{}{
  \begin{minipage}{.05\textwidth}
    \includegraphics[height=\textwidth]{qr-code.pdf}
  \end{minipage}
  \begin{minipage}{.55\textwidth}
    Check out the full paper at \texttt{http://tiny.cc/f6x6gz}\\
    or scan the QR code on the left! \\
    % Inspired by \texttt{\#betterposter} by Mike Morrison
    {Email: \texttt{niklas.heim@aic.fel.cvut.cz}}
  \end{minipage}
}
\column{0.3}
\block{}{
  \centering
  \begin{minipage}{.04\textwidth}
    \includegraphics[height=\textwidth]{aic-logo.png}
  \end{minipage}
  \hspace{.04\textwidth}
  \begin{minipage}{.05\textwidth}
    \includegraphics[height=\textwidth]{cvut-logo.jpeg}
  \end{minipage}
  \hspace{.04\textwidth}
}


\end{columns}

\end{document}
